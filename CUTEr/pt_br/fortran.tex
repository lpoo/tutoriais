\subsection{Exemplos em Fortran}
Fizemos uma implementação do método de máxima descida. Temos dois arquivos na implementação:
\begin{itemize}
 \item \verb+gradient.f+: Este arquivo contém a definição do método.
 \item \verb+gradientmain.f+: Este arquivo contém a rotina principal do fortran.
\end{itemize}
Além desses arquivos também é necessário um arquivo com as definições das subrotinas
\begin{itemize}
 \item \verb+inip(n,x)+: Retorna n, a dimensão do problema, e x, o ponto inicial.
 \item \verb+evalf(n,x,f)+: Recebe a dimensão do problema n, e o ponto x, e retorna o
valor da função objetivo em f.
 \item \verb+evalg(n,x,g)+: Recebe a dimensão do problema n, e o ponto x, e retorna o
valor do gradiente da função objetivo em g.
 \item \verb+endp(n,x)+: Imprime informações sobre a solução.
\end{itemize}
Para criar a interface em fortran, é necessário apenas criar um arquivo com as subrotinas
acima.
O arquivo com a interface (sem os comentários está abaixo:)
\begin{code}{fortran}
      subroutine inip(n,x)

      implicit none

      integer n

      double precision x(*)
      
      integer i

      integer err, ifile, nt, m, nmax
      PARAMETER (nmax=10000)
      double precision bl(nmax), bu(nmax)

      ifile = 30
      OPEN(ifile, FILE='OUTSDIF.d', FORM='FORMATTED',
     $     STATUS='OLD',IOSTAT=err)
      REWIND ifile
      IF (err.NE.0) THEN
        WRITE(*,*)'Could not open the OUTSDIF.d file'
        STOP
      ENDIF

      CALL cdimen(ifile, nt, m)

      if (nt.GT.nmax) THEN
        WRITE(*,*)'Increase nmax'
        STOP
      ENDIF
      if (m.GT.0) THEN
        WRITE(*,*)'Cannot handle constraints' 
        STOP
      ENDIF

      CALL usetup(ifile, 7, n, x, bl, bu, nmax)

      DO i = 1,n
        IF ((bl(i).GT.-1.0D20).OR.(bu(i).LT.1.0D20)) THEN
          WRITE(*,*)'Cannot handle boxes'
          STOP
        ENDIF
      ENDDO
      
      end
      
C     ******************************************************************
C     ******************************************************************

      subroutine evalf(n,x,f)

      implicit none

      integer n
      double precision f

      double precision x(n)

      CALL ufn(n, x, f)
          
      end

C     ******************************************************************
C     ******************************************************************

      subroutine evalg(n,x,g)

      implicit none

      integer n

      double precision g(n),x(n)
      
      CALL UGR(n, x, g)

      end 

C     ******************************************************************
C     ******************************************************************

      subroutine endp(n,x)

      implicit none

      integer n
      double precision x(n)
      integer i
      
      write(*,*)'Solution:'
      do i = 1,n
          write(*,*)x(i)
      end do
      
      end 
\end{code}
