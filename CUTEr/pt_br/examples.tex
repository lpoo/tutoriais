\section{Exemplos}
Como exemplos do CUTEr, vamos criar bibliotecas em diversas linguagens para resolver o 
problema de minimização irrestrita. 
Vamos implementar o método de máxima descida com busca linear utilizando o
critério de Armijo.

Considere o problema
\begin{equation*}
 \min f(x), \qquad x \in \Rn{n},
\end{equation*}
onde $f:\RnemR{n}$ é contínua e derivável. Vamos procurar um ponto estacionário para esse
problema, isto é, um ponto $x^* \in \Rn{n}$ tal que $$\nabla f(x^*) = 0.$$
Obviamente, como vamos implementar este método, vamos parar quando encontrarmos um
iterando $x^k$ tal que $\norma{\nabla f(x^k)} \leq \varepsilon,$ onde $\varepsilon > 0$ é
dado. O método está descrito a seguir.
\begin{algorithm}[H]
\caption{Método de Máxima Descida}
 \begin{algorithmic}[1]
 \State Dados $x^0 \in \Rn{n}$, $\varepsilon > 0$, $\alpha \in (0,1)$, $k = 0$.
 \While{$\norma{\nabla f(x^k)} > \varepsilon$}
  \State $d^k = -\nabla f(x^k)$
  \State $\lambda_k = 1$
  \While {$f(x^k + \lambda_kd^k) > f(x^k) + \alpha \lambda_k \nabla f(x^k)^Td^k$}
   \State $\lambda_k = \lambda_k/2$
  \EndWhile
  \State $x^{k+1} = x^k + \lambda_kd^k$
  \State $k = k + 1$
 \EndWhile
 \State $x^* = x^k$.
 \end{algorithmic}
\end{algorithm}
Vamos implementar este método em algumas linguagens, e às vezes, mais de uma vezes, para
exemplificar a interface CUTEr.

\subsection{Instruções Gerais}

\emph{
Os passos a seguir são feitos normalmente após a criação da biblioteca, porque
primeiro você precisa fazer com que ela funcione em outros problemas. Na dúvida,
faça um dos exemplos a seguir e depois volte para ler esta seção.}

Para que seu programa funcione com o CUTEr, você irá precisar de um arquivo de 
protótipo que pode ser copiado de \verb+$CUTER/build/prototypes/gen90.sh.pro+.
Nesse arquivo, você irá mudar algumas variáveis para o compilador do CUTEr
encontrar seus arquivos. As linhas sem comentários estão mostradas a seguir:
\begin{code}{bash}
export PACK=gen90
export PACKAGE=gen
export PACK_PRECISION="single double"
export PACKOBJS="cuterinter.o gen90.o"
export PACKLIBS=""
export SPECS="GEN.SPC"
\end{code}
Vamos supor que temos um pacote chamado \verb+cutertest+, e que esse pacote precisa
de um objeto \verb+auxiliar.o+, e de uma biblioteca do sistema \verb+libalgo.a+.
Além disso, você deve ter criado (veja nos exemplos) um arquivo específico para
fazer a interface entre o CUTEr e o pacote \verb+cutertest+. Vamos supor que
a linguagem em questão seja \verb+fortran+, então você deve ter criado um
arquivo \verb+cutertestmain.f+. A variável \verb+PACK+ define um pequeno
nome para o seu pacote, e \verb+PACKAGE+ é o nome de seu pacote.
\verb+PACK_PRECISION+ é a precisão do seu pacote (muitas vezes double).
\verb+PACKOBJS+ são os objetos que você deve colocar na pasta
\verb+$MYCUTER/(precision)/bin+ onde (precision) é a precisão que você instalou
o CUTEr.
\verb+PACKLIBS+ são as bibliotecas que o pacote precisa, e \verb+SPECS+ se refere
a algum arquivo de especificações que o seu pacote pode depender. Esse arquivo pode
ser local (isto é, na pasta onde você irá chamar o CUTEr) ou na pasta
\verb+$CUTER/common/src/pkg/$PACKAGE/+.

Para o nosso exemplo deveríamos ter no arquivo chamado \verb+cutertest.sh.pro+
as seguintes definições:
\begin{code}{bash}
export PACK=cutertest
export PACKAGE=cutertest
export PACK_PRECISION="double"
export PACKOBJS="auxiliar.o"
export PACKLIBS="-lalgo"
export SPECS=""
\end{code}
Note que devemos copiar o arquivo \verb+auxiliar.o+ para \verb+$MYCUTER/double/bin+.
Agora, precisamos passar esse arquivo por um formatador de texto utilizando um
script que o CUTEr fornece e devemos habilitá-lo para execução. Para isso, os comandos são
\begin{terminal}
$ sed -f $MYCUTER/double/config/script.sed cutertest.sh.pro > $MYCUTER/bin/cutertest
$ chmod a+x $MYCUTER/bin/cutertest
\end{terminal}
Agora passamos o arquivo de interface e o arquivo compilado para o CUTEr.
\begin{terminal}
$ cp -f cutertestmain.f $MYCUTER/double/bin/cutertestma.f
$ cp -f cutertestmain.o $MYCUTER/double/bin/cutertestma.o
\end{terminal}
Tendo feito isso, agora seu arquivo está pronto para ser executado. Use o comando
\begin{terminal}
$ runcuter -p cutertest -D ROSENBR
\end{terminal}

\subsection{Exemplos em Fortran}
Fizemos uma implementação do método de máxima descida. Temos dois arquivos na implementação:
\begin{itemize}
 \item \verb+gradient.f+: Este arquivo contém a definição do método.
 \item \verb+gradientmain.f+: Este arquivo contém a rotina principal do fortran.
\end{itemize}
Além desses arquivos também é necessário um arquivo com as definições das subrotinas
\begin{itemize}
 \item \verb+inip(n,x)+: Retorna n, a dimensão do problema, e x, o ponto inicial.
 \item \verb+evalf(n,x,f)+: Recebe a dimensão do problema n, e o ponto x, e retorna o
valor da função objetivo em f.
 \item \verb+evalg(n,x,g)+: Recebe a dimensão do problema n, e o ponto x, e retorna o
valor do gradiente da função objetivo em g.
 \item \verb+endp(n,x)+: Imprime informações sobre a solução.
\end{itemize}
Para criar a interface em fortran, é necessário apenas criar um arquivo com as subrotinas
acima.
O arquivo com a interface (sem os comentários está abaixo:)
\begin{code}{fortran}
      subroutine inip(n,x)

      implicit none

      integer n

      double precision x(*)
      
      integer i

      integer err, ifile, nt, m, nmax
      PARAMETER (nmax=10000)
      double precision bl(nmax), bu(nmax)

      ifile = 30
      OPEN(ifile, FILE='OUTSDIF.d', FORM='FORMATTED',
     $     STATUS='OLD',IOSTAT=err)
      REWIND ifile
      IF (err.NE.0) THEN
        WRITE(*,*)'Could not open the OUTSDIF.d file'
        STOP
      ENDIF

      CALL cdimen(ifile, nt, m)

      if (nt.GT.nmax) THEN
        WRITE(*,*)'Increase nmax'
        STOP
      ENDIF
      if (m.GT.0) THEN
        WRITE(*,*)'Cannot handle constraints' 
        STOP
      ENDIF

      CALL usetup(ifile, 7, n, x, bl, bu, nmax)

      DO i = 1,n
        IF ((bl(i).GT.-1.0D20).OR.(bu(i).LT.1.0D20)) THEN
          WRITE(*,*)'Cannot handle boxes'
          STOP
        ENDIF
      ENDDO
      
      end
      
C     ******************************************************************
C     ******************************************************************

      subroutine evalf(n,x,f)

      implicit none

      integer n
      double precision f

      double precision x(n)

      CALL ufn(n, x, f)
          
      end

C     ******************************************************************
C     ******************************************************************

      subroutine evalg(n,x,g)

      implicit none

      integer n

      double precision g(n),x(n)
      
      CALL UGR(n, x, g)

      end 

C     ******************************************************************
C     ******************************************************************

      subroutine endp(n,x)

      implicit none

      integer n
      double precision x(n)
      integer i
      
      write(*,*)'Solution:'
      do i = 1,n
          write(*,*)x(i)
      end do
      
      end 
\end{code}

\subsection{Exemplos em C}
Fizemos três implementações do método de máxima descida. A primeira é uma implementação
que não leva me conta o CUTEr, e depois adapta o CUTEr para o problema. A segunda já leva
em conta o formato das funções do CUTEr e faz pouca adaptação posteriormente. A terceira
usa exatamente as funções do CUTEr, não necessitando de adaptação.

Cada implementação do método de máxima descida consiste de dois arquivos:
\verb+steepest_descent.h+ e \verb+steepest_descent.c+. No .h, definimos que funções
iremos chamar, e uma estrutura com as informações da execução. As funções para a 
primeira implementação são
\begin{itemize}
 \item \verb+double Norm (double * x, unsigned int n);+ \\
Esta função calcula a norma 2 de um vetor \verb+x+ com tamanho \verb+n+.
 \item \verb+double NormSqr (double * x, unsigned int n);+ \\
Esta função calcula o quadrado da norma 2 de um vetor \verb+x+ com tamanho \verb+n+. É mais
rápido que a função \verb+Norm+ pois não envolve raiz quadrada.
 \item \verb+SteepestDescent (double * x, unsigned int n, Status * status);+ \\
Esta é a função que encontra o ponto estacionário. \verb+x+ entra como ponto inicial e 
sai como a solução. \verb+n+ é a dimensão do problema e \verb+status+ é um ponteiro para
a estrutura de informações.
 \item \verb+SD_Print (double * x, unsigned int n, Status * status);+ \\
Esta função imprime o vetor \verb+x+ e as informações da execução do problema.
\end{itemize}
A estrutura do problema
\begin{code}{C}
typedef struct _Status {
  unsigned int iter;
  double f, ng;
  unsigned int n_objfun, n_gradfun;
} Status;
\end{code}
\verb+iter+ é o número de iterações que o algoritmo executou, \verb+f+ é o valor da
função objetivo na solução, \verb+ng+ é a norma do gradiente da função objetivo,
\verb+n_objfun+ é o número de cálculos da função objetivo e \verb+n_gradfun+ é o número
de cálculos do gradiente. Mostraremos as diferenças das outras implementações posteriormente.

Nosso arquivo .c contém as definições das funções acima, e contém uma declaração de função
usada para acessar a função objetivo e o gradiente. Na primeira implementação, essa declaração
é
\begin{code}{C}
double objfun  (double * x, unsigned int n);
void   gradfun (double * x, unsigned int n, double * g);
\end{code}
As funções \verb+Norm+, \verb+NormSqr+ e \verb+SD_print+ são idênticas para todas as
implementações e serão deixadas de fora. A implementação do método em si encontra-se abaixo.
\scriptsize
\begin{code}{C}
void SteepestDescent (double * x, uint n, Status *status) { 
  double * g, f, fp;
  double * xp, lambda, ng_sqr;
  uint i;

  if ( (x == 0) || (status == 0) )
    return;

  g  = (double *) malloc(n * sizeof(double) );
  xp = (double *) malloc(n * sizeof(double) );
  status->iter = 0;
  status->n_objfun = 0;
  status->n_gradfun = 0;

  f = objfun(x, n);
  status->n_objfun++;
  gradfun(x, n, g);
  status->n_gradfun++;

  status->ng = Norm(g, n);

  while (status->ng > EPSILON) {
    lambda = 1;

    for (i = 0; i < n; i++) {
      xp[i] = x[i] - g[i];
    }

    fp = objfun(xp, n);
    status->n_objfun++;
    ng_sqr = status->ng*status->ng;

    while (fp > f - 0.5 * lambda * ng_sqr) {
      for (i = 0; i < n; i++) {
        xp[i] = x[i] - lambda*g[i];
      }
      lambda = lambda/2;
      fp = objfun(xp, n);
      status->n_objfun++;
    }

    for (i = 0; i < n; i++)
      x[i] = xp[i];

    f = objfun(x, n);
    status->n_objfun++;
    gradfun(x, n, g);
    status->n_gradfun++;
    status->ng = Norm(g, n);
    status->iter++;
  }

  status->f = f;

  free(xp);
  free(g);
}
\end{code}
\normalsize
Um código de exemplo para esse teste é
\scriptsize
\begin{code}{C}
#include <stdio.h>
#include "steepest_descent.h"

/*
 * Each file testx.c is a different problem. The user will have to
 * implement his own file, defining objfun and gradfun.
 */

/*
 * This problem is
 *
 *  min f(x) = 0.5*(x_1^2 + x_2^2)
 *
 *  starting from point x0 = (1,2);
 *
 */

double objfun (double * x, unsigned int n) {
  (void)n;
  return 0.5 * (x[0]*x[0] + x[1]*x[1]);
}

void gradfun (double * x, unsigned int n, double * g) {
  unsigned int i;

  for (i = 0; i < n; i++)
    g[i] = x[i];
}

int main () {
  double x[2];
  Status status;

  x[0] = 1;
  x[1] = 2;

  SteepestDescent(x, 2, &status);

  SD_Print(x, 2, &status);

  return 0;
}
\end{code}
\normalsize
Este código implementa o problema de minimizar $f(x) = \meio\norma{x}^2$ em duas dimensões.
Note como temos que declarar as funções \verb+objfun+ e \verb+gradfun+. Sem elas teríamos
erros na compilação. Veja os arquivos \verb+test2.c+ e \verb+test3.c+
para outros exemplos.

A interface para o CUTEr é o arquivo \verb+c_example1main.c+:
\scriptsize
\begin{code}{C}
#include "cuter.h"
#include "CExample1/steepest_descent.h"

double objfun (double * x, unsigned int n) {
  double F = 0;
  int N = n;
  UFN(&N, x, &F);
  return F;
}

void gradfun (double * x, unsigned int n, double * g) {
  int N = n;
  UGR(&N, x, g);
}

int MAINENTRY () {
  double *x, *bl, *bu;
  char fname[10] = "OUTSDIF.d";
  int nvar = 0, ncon = 0, nmax;
  int funit = 42, ierr = 0, fout = 6;
  int i;
  Status status;

  FORTRAN_OPEN(&funit, fname, &ierr);
  CDIMEN(&funit, &nvar, &ncon);

  if (ncon > 0) {
    printf("ERROR: Problem is not unconstrained\n");
    return 1;
  }

  x  = (double *) malloc (sizeof(double) * nvar);
  bl = (double *) malloc (sizeof(double) * nvar);
  bu = (double *) malloc (sizeof(double) * nvar);

  USETUP(&funit, &fout, &nvar, x, bl, bu, &nmax);

  for (i = 0; i < nvar; i++) {
    if ( (bl[i] > -CUTE_INF) || (bu[i] < CUTE_INF) ) {
      printf("ERROR: Problem has bounds\n");
      return 1;
    }
  }

  SteepestDescent(x, nvar, &status);

  SD_Print(x, nvar, &status);

  free(x);
  free(bl);
  free(bu);

  return 0;
}
\end{code}
\normalsize
Note que também é necessário definir as funções \verb+objfun+ e \verb+gradfun+. Essas
funções, por sua vez, chamam as funções correspondentes em CUTEr para esse serviço.
A função \verb+UFN+ calcula o valor da função objetivo e a função \verb+UGR+ calcula
o valor do gradiente. Note que quando compilamos uma função em Fortran, ele recebe um
nome em letras minúsculas e com um \_ (underline) na frente (no caso do gfortran. Outros
compiladores podem divergir). O arquivo \verb+cuter.h+ define macros para todas as funções
do CUTEr serem chamados com letras maiúsculas. Note ainda que a função \verb+UFN+ recebe
um ponteiro para \verb+int+ e dois ponteiros para \verb+double+, sendo o primeiro para o
vetor $x$ e o segundo para o valor da função objetivo. As funções do CUTEr para C recebem
ponteiros em todos os valores. Os valores que são vetores não precisam de um ponteiro 
adicional. Note também que como utilizamos \verb+unsigned int+ para os tamanhos, tivemos
que converter os valores para \verb+int+.

Veja agora o Exemplo 2. Nesse exemplo consideramos que as funções devem estar no mesmo
formato que a função do CUTEr.
\begin{code}{C}
void ufn (int * n, double * x, double * f);
void uofg (int * n, double * x, double * f, double * g, long int * grad);
\end{code}
O nome das funções foram escolhidas para seguir exatamente o formato do CUTEr, mas aqui
elas poderiam ser qualquer coisa. Note que a função \verb+uofg+ foi utilizada no lugar
da função \verb+ugr+. Essa função já calcula a função objetivo e o gradiente, sendo mais
rápida que chamadas individuais. Com essa mudança, esse programa é levemente mais rápido
que o outro. O resto do arquivo foi mudado de acordo a seguir essas mudanças.

A interface também teve uma mudança na definição das funções:
\begin{code}{C}
void ufn (int * n, double * x, double * f) {
  UFN(n, x, f);
}

void uofg (int * n, double * x, double * f, double * g, long int * grad) {
  UOFG(n, x, f, g, grad);
}
\end{code}
Essa interface fica muito mais natural de ser utilizada num problema com CUTEr. Não precisamos
converter nenhuma variável, simplesmente fazer a chamada da função com os parâmetros já dados.
Note, no entanto, que estamos utilizando \verb+long int+ para as variáveis lógicas do CUTEr.
Essa é uma definição do arquivo \verb+cuter.h+. Se mudarmos essa definição, então devemos
criar uma variável \verb+logical GRAD = *grad+ e chamar a função com
\verb+UOFG(n, x, f, g, &GRAD)+.

A última interface já declara as funções que o CUTEr irá definir. Então no arquivo .c temos
\begin{code}{C}
void ufn_ (int * n, double * x, double * f);
void uofg_ (int * n, double * x, double * f, double * g, long int * grad);
\end{code}
e no arquivo da interface não temos nenhuma definição de função. Lembre-se que o Fortran
compilado com gfortran cria as funções com minúsculas e \_ na frente. Se mudarmos o
compilador pode não funcionar. 
Além disso, a definição dessas funções é, na verdade
\begin{code}{Fortran}
void UFN( integer *n, doublereal *x, doublereal *f );
void UOFG( integer *n, doublereal *x, doublereal *f, doublereal *g,
      logical *grad);
\end{code}
e esses tipos são definidos no arquivo \verb+cuter.h+, podendo ser alterados pelo usuário.
Se isso acontecer, é necessário mudar todo o programa.

Uma alternativa é utilizar \verb+typedef+s para definir os tipos próprios, e deixar o
acesso desses tipos para o usuário (assim como o arquivo \verb+cuter.h+). Dessa maneira,
se o usuário tiver necessidade de mudar o arquivo \verb+cuter.h+, ele também pode (deve)
mudar o arquivo com esses \verb+typedef+s.

Para compilar a interface CUTEr dos testes utilize
\begin{terminal}
$ make cuter
\end{terminal}
Para rodar os testes utilize o comando
\begin{terminal}
$ runcuter -p c_example# -D BARD
\end{terminal}
onde \# é o número do exemplo e BARD é um dos problemas em que esse exemplo converge. Note
que é preferível criar uma pasta separada para rodar os testes, já que eles geram lixo na
pasta.

\input{c++}
\input{python}
\subsection{Exemplos em MATLAB}
Com o {\tt CUTEr} devidamente instalado, vejamos como ocorre a interface com o {\tt MATLAB}.

Primeiro, crie uma pasta {\tt Test}, onde ser\~ao gerados os arquivos decodificados do problema
 e o arquivo {\tt .mex} da interface para {\tt MATLAB}.

No terminal, dentro da pasta {\tt Test}, execute o comando:
\begin{terminal}
$ runcuter -p mx -D  ROSENBR
\end{terminal}
Al\'em dos arquivos padr\~ao, gerados pelo decoder para o problema {\tt ROSENBR}, haver\'a 
tamb\'em o arquivo: {\tt mcuter.mex}. A extens\~ao {\tt .mex} poder\'a variar conforme a 
instala\c{c}\~ao do {\tt CUTEr} e o sistema operacional.

Em seguida, v\'a para o {\tt MATLAB}. Adicione as pastas {\tt \$CUTER/common/src/matlab} e {\tt 
\$MYCUTER/bin} ao {\tt path} do {\tt MATLAB}. Isso pode ser feito atrav\'es do menu {\tt File\ 
>\ Set Path...} ou pelo prompt do {\tt MATLAB} usando o comando {\tt addpath}.

Dentro do {\tt MATLAB}, v\'a para a pasta {\tt Test}. Para verificar se est\'a tudo certo, 
execute o comando:
\begin{code}{matlab}
>> prob = cuter_setup()

prob = 

         n: 2
         m: 0
      nnzh: 3
      nnzj: 0
         x: [2x1 double]
        bl: [2x1 double]
        bu: [2x1 double]
         v: [0x1 double]
        cl: [0x1 double]
        cu: [0x1 double]
    equatn: [0x1 logical]
    linear: [0x1 logical]
      name: 'ROSENBR   '
\end{code}
Se estiver tudo correto, a sa\'ida dever\'a ser como acima.

O comando {\tt cuter\_setup}, que inicializa o problema e fornece suas carcacter\'isticas, se 
encontra na pasta {\tt \$CUTER/src/common/matlab}, dentro da qual est\~ao as demais rotinas 
(arquivos {\tt .m}) para acessar fun\c{c}\~ao objetivo, gradiente, restri\c{c}\~oes, etc.

Por exemplo, para saber o valor da fun\c{c}\~ao objetivo em um determinado ponto, usamos:
\begin{code}{matlab}
>> f = cuter_obj([-1 2]')

f =

   104
\end{code}
Abaixo temos um c\'odigo em {\tt MATLAB} que implementa o m\'etodo do gradiente. Para acessar a 
fun\c{c}\~ao objetivo e o gradiente, fizemos uso da fun\c{c}\~ao {\tt cuter\_obj}.
\begin{code}{matlab}
function [x,f,gradnorm,iter,flag] = cute_gradient()
%
%   [x,f,gradnorm,iter,flag] = cute_gradient()
%
%   flags:
%   -2      maximum number of iterations reached
%   -1      too short step size
%    1      gradient norm less than eps0
%

    % inicializando o problema
    prob = cuter_setup();
    
    % dimensao do problema
    n = prob.n;
    
    % ponto inicial
    x0 = prob.x;
    
    %====================================
    % Gradient method with linesearch
    %====================================
    maxit=n*10000;
    gamma=1e-4;
    eps0=1e-5;
    tmin=1e-8;
    done=0;
    flag=0;
    k=0;
    x=x0;
    
    while ~done
        k=k+1;
        
        % maximum number of iterations test
        if k>maxit
           done=1;
           flag=-2;
           continue
        end
        
        x0=x;
        
        [f,g] = cuter_obj(x);
        f0=f;
        
        gradnorm=norm(g,2);
        
        % gradient norm test
        if gradnorm<eps0
            done=1;
            flag=1;
            continue
        end
        
        % search direction
        d = -g;
        gtd = g'*d;
        t = 1;
        
        x = x0 + t*d;
        f = cuter_obj(x);
        
        % Armijo linesearch
        while f > f0 + t*gamma*gtd
            t = 0.5*t;
            if t<tmin
                done=1;
                flag=-1;
            end
            
            x = x0 + t*d;
            f = cuter_obj(x);
        end
        
    end
    
    iter=k;

end
\end{code}
E executando o c\'odigo acima, obtemos
\begin{code}{matlab}
>> [x,f,gradnorm,iter,flag] = cute_gradient()

x =

    1.0000
    1.0000


f =

   6.1319e-11


gradnorm =

   9.9971e-06


iter =

       10917


flag =

     1
\end{code}

\'E poss\'ivel tamb\'em utilizar as rotinas de otimiza\c{c}\~ao do pr\'oprio {\tt MATLAB}. Por 
exemplo, para utilizar o {\tt fminunc}, usamos:
\begin{code}{matlab}
>> prob = cuter_setup();
>> [x,f,flag] = fminunc(@(x) cuter_obj(x),prob.x)
Warning: Gradient must be provided for trust-region method;
  using line-search method instead. 
> In fminunc at 356

Local minimum found.

Optimization completed because the size of the gradient is less than
the default value of the function tolerance.

<stopping criteria details>


x =

    1.0000
    1.0000


f =

   2.8336e-11


flag =

     1
\end{code}
Lembrando que as op\c{c}\~oes do solver s\~ao ajustadas pelo comando {\tt optimset}.
\begin{code}{matlab}
>> prob = cuter_setup();
>> opts = optimset('GradObj','on');
>> [x,f,flag] = fminunc(@(x) cuter_obj(x),prob.x,opts)

Local minimum possible.

fminunc stopped because the final change in function value relative to 
its initial value is less than the default value of the function tolerance.

<stopping criteria details>


x =

    1.0000
    1.0000


f =

   4.0035e-13


flag =

     3
\end{code}

