%!TEX root=cutest.tex
\subsection{Exemplos em Fortran}
Fizemos uma implementação do método de máxima descida. Temos dois arquivos na
implementação:
\begin{itemize}
 \item \verb+gradient.f+: Este arquivo contém a definição do método.
 \item \verb+gradientmain.f+: Este arquivo contém a rotina principal do fortran.
\end{itemize} Além desses arquivos também é necessário um arquivo com as
definições das subrotinas
\begin{itemize}
 \item \verb+inip(n,x)+: Retorna n, a dimensão do problema, e x, o ponto
 inicial.
 \item \verb+evalf(n,x,f)+: Recebe a dimensão do problema n, e o ponto x, e
retorna o valor da função objetivo em f.
 \item \verb+evalg(n,x,g)+: Recebe a dimensão do problema n, e o ponto x, e
retorna o valor do gradiente da função objetivo em g.
 \item \verb+endp(n,x)+: Imprime informações sobre a solução.
\end{itemize}
Para criar a interface em fortran, é necessário apenas criar um arquivo com as
subrotinas acima. O arquivo com a interface (sem os comentários está abaixo:)
\lstinputlisting[style=codestyle, language=fortran]
{../Examples/FortranExample1/fortran_examplemain.f}
